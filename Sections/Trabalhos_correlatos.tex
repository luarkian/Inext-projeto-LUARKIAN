\label{chapter:correlatos}
Este capítulo tem como objetivo apresentar os principais trabalhos relacionados com o sistema proposto neste trabalho, exaltando as diferenças e contribuições para o método proposto neste trabalho. Os trabalhos em específico abordados utilizam tag RFID ou referenciam sistemas que fazem uso de tal ferramenta para localizar e identificar objetos em ambientes fechados: An indoor localization mechanism using active RFID tag \cite{mechanismRFID2006}, Real-Time Locating Systems Using Active RFID for Internet of Things \cite{realtimeRFID2016}, Object localization using RFID \cite{localization2010} e RFID localization algorithms and applications-A review \cite{rfid2009review}.

\section{An indoor localization mechanism using active RFID tag}
O trabalho de \citeauthor{mechanismRFID2006} faz uma análise do método utilizado no sistema LANDMARC e depois propõe uma melhoria para o método. O LANDMARC é um sistema de localização que faz uso de tag de referência. As tags de referência são tags ativas colocadas em pontos fixos que servem como pontos de referência para o sistema e calibração da localização, esse sistema utiliza tags ativas visando diminuir a quantidade de leitores RFID para baratear o sistema.

%Funcionamento do landmarc
\par
Segundo \citeauthor{mechanismRFID2006} o LANDMARC funciona da seguinte maneira: primeiro os leitores detectam as tags de referências que estão no seu alcance, após isso é concedido ponderação as tags de referências de acordo com RSS, sendo valor mais alto para as mais próximas. O passo seguinte é realizar o cálculo da distância euclidiana na RSS entre uma tag de rastreamento e uma tag de referência e quanto menor o resultado mais próximo da tag de referência. Por fim o sistema escolhe k tags de referências com os menores valores no resultado do cálculo e obtém a localização da tag de rastreamento por meio da $(x,y) = \sum^k_{i=1}w_i(x_i,y_i), w_j=\frac{\frac{1}{E_i^2}}{\sum_{i=1}^k\frac{1}{E^2_i}}$. 
%Melhoria proposta
\par
\citeauthor{mechanismRFID2006} mostra que LANDMARC possui alguns problemas, sendo efetuando cálculos desnecessários durante a escolha das tags vizinhas e comparação de valores RSS de leitores que não estão no alcance da tag alvo. Para melhorar o sistema \citeauthor{mechanismRFID2006} propõe a utilização apenas dos leitores RFID  e tags de referência que alcançam o alvo, sendo que o número de leitores e das tags de referência tem que ser maior que 3, pois depois o método consiste em utilizar a técnica de Triangulação.
%Comparação com o método proposto
\par
O método utilizado por \citeauthor{mechanismRFID2006} tem objetivo de ter alcançar grande precisão no posicionamento em relação ao ambiente, já o método proposto neste trabalho tem objetivo de saber exatamente a sala que os objetos estão, tendo em vista que o ambiente possui inúmeras salas. Entretanto essa aplicação para posicionamento na sala pode ser acrescentado e estudada em trabalhos futuros.

\section{Real-Time Locating Systems Using Active RFID for Internet of Things}
O trabalho de \citeauthor{realtimeRFID2016} propõe um sistema de localização em tempo real, o iLocate é mais um sistema que faz uso de tags ativas, e  utilizar-se de ZigBee para assim assegurar que a transmissão de dados a longa distância seja garantida. 
\par
O sistema proposto por \citeauthor{realtimeRFID2016} inicialmente aplica uma técnica de \textit{fingerprinting} que seria uma técnica equivalente ao reconhecimento de padrões abordada em \autoref{subsection:wlan}, iniciando com uma leituras breve de RSSI e implantação de tags de referências no mapa de interesse, seguindo para o armazenamento das leituras no banco de dados e construção de uma matriz com dados de RSSI. 
\par 
O iLocate conta com um refinamento para ter uma localização mais precisa, que funciona criando tags de referência virtuais que também utiliza a técnica de \textit{fingerprinting} para comparar com a reais. Outro recurso encontrado no iLocale é a comunicação tag-tag em que possibilita a troca de mensagens entre tags de referências que estão no mesmo intervalo de trabalho \cite{realtimeRFID2016}.
\par
A utilização de uma base de dados por \citeauthor{realtimeRFID2016} é um recurso que também será utilizado no sistema proposto neste trabalho, por mais que os dados armazenados sejam diferentes o banco de dados terá função importante.

\section{Object localization using RFID}
Na pesquisa de \citeauthor{localization2010} é proposto um método de localização que permite estimar a posição de objetos com rapidez e velocidade utilizando a variação dos níveis de potência dos leitores, o método ainda faz uso de tags de referência para auxiliar na localização das tags que serão rastreadas.

\par
O método proposto por \citeauthor{localization2010} foi aplicado em uma sala cuja as dimensões são 2m x 3m. A região que assemelha a um retângulo foi dividida em oito sub-regiões de tamanho iguais e denominadas de setores, em seguida cada setor foi dividido em quatro regiões de tamanho iguais chamados de quadrantes.

\par
Depois de fazer a divisão do ambiente é passado para inserção de tags de referências em cada quadrante, essas tags de referência são utilizadas inicialmente para calibrar os leitores RFID com uma relação de potência em contraste com distância. Por fim é utilizados algoritmos que ficam variando o nível de potência do leitor, esses algoritmos podem iniciar a variação da potência do menor para maior até alcançar um nível mínimo para localizar o objeto ou podendo utilizar uma variação da potência do maior para o menor \citeauthor{localization2010}.

\par
O método \citeauthor{localization2010} diferencia-se do método proposto no caso de buscar uma localização e posicionamento preciso em ambiente fechado, o que não se torna uma prioridade para o método proposto neste trabalho.


\section{RFID localization algorithms and applications-A review}
O artigo de \citeauthor{rfid2009review} tem uma abordagem diferente dos trabalhos anteriores, não propondo nenhum método de localização mas mostrando as aplicações existentes, revisando algoritmos de localização indoor e potenciais da localização de RFID.
\par
RFID pode ser empregado em várias aplicações, de acordo com \citeauthor{rfid2009review} RFID é utilizado no porto de Cingapura para posicionar os contêiner, na saúde para localização de pacientes, suprimentos e equipamentos  e também na gestão de material de construção permitindo monitoramento do andamento e status da construção.
\par
No artigo é realizado uma revisão dos algoritmos de localização classificando-os em dois grupos, o grupo que calibra a distribuição do sinal de radiofrequência e depois estima a posição do objeto e o segundo grupo que calculam diretamente a posição do objeto utilizando dados de RSSI. Dentro do primeiro grupo encontramos os algoritmo de multilateração e inferência bayesiana, já no segundo grupo estão os algoritmos de aprendizados em proximidade e de proximidade.
\par
O artigo de \citeauthor{rfid2009review} teve grande contruibuição para esta pesquisa, solucionando duvidas e abordando assuntos importates para utilização de RFID e sistemas de localização.
