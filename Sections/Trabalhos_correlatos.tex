
\label{chapter:correlatos}

Este capítulo tem como objetivo apresentar os principais trabalhos relacionados com o sistema proposto neste trabalho, 
exaltando as diferenças e contribuições para o método proposto neste trabalho. Os trabalhos em específico abordados 
utilizam RFID ou referenciam sistemas que fazem uso de tal ferramenta para localizar e identificar objetos em ambientes 
fechados.
%: An indoor localization mechanism using active RFID tag \cite{mechanismRFID2006}, 
%Real-Time Locating Systems Using Active RFID for Internet of Things \cite{realtimeRFID2016}, 
%Object localization using RFID \cite{localization2010} e 
%RFID localization algorithms and applications-A review \cite{rfid2009review}.

%
%
\section{\textit{An indoor localization mechanism using active RFID tag}}

O trabalho de \citeonline{mechanismRFID2006} faz uma análise do método utilizado no sistema LANDMARC 
%e depois propõe uma melhoria para o método. O LANDMARC 
que é um sistema de localização por meio de tag RFID. As tags de referência são tags ativas colocadas\todo{No objeto?} em pontos fixos 
que servem como pontos de referência para o sistema e calibração da localização, esse sistema utiliza tags ativas visando 
diminuir a quantidade de leitores RFID para baratear o sistema.

%Funcionamento do landmarc
Segundo \citeonline{mechanismRFID2006} o LANDMARC funciona da seguinte maneira: primeiro os leitores detectam as tags RFID que estão 
no seu alcance, após isso é concedido ponderação as tags de referências de acordo com RSS, sendo valor mais alto para as mais 
próximas. O passo seguinte é realizar o cálculo da distância euclidiana na RSS entre uma tag de rastreamento e uma tag de 
referência e quanto menor o resultado mais próximo da tag de referência. Por fim o sistema escolhe $k$ tags com os menores valores no 
resultado do cálculo e obtém a localização da tag de rastreamento por meio da seguinte fórmula.

\begin{equation}
(x,y) = \sum^k_{i=1}w_i(x_i,y_i), w_j=\frac{\frac{1}{E_i^2}}{\sum_{i=1}^k\frac{1}{E^2_i}} 
\end{equation}
%Melhoria proposta


\citeonline{mechanismRFID2006} mostra que LANDMARC possui alguns problemas, sendo efetuando cálculos desnecessários 
durante a escolha das tags vizinhas e comparação de valores RSS de leitores que não estão no alcance da tag alvo. 
Para melhorar o sistema, \citeonline{mechanismRFID2006} propõe a utilização apenas dos leitores RFID  e tags de referência 
que alcançam o alvo, sendo que o número de leitores e das tags de referência tem que ser maior que 
$3$, pois depois o método consiste em utilizar a técnica de Triangulação.
%Comparação com o método proposto


O método utilizado por \citeonline{mechanismRFID2006} tem objetivo de ter alcançar grande precisão no 
posicionamento em relação ao ambiente, já o método proposto neste trabalho tem objetivo de saber exatamente a 
sala que os objetos estão, tendo em vista que o ambiente possui inúmeras salas. Entretanto essa aplicação para 
posicionamento na sala pode ser acrescentado e estudada em trabalhos futuros, visto que neste primeiro momento 
do sistema proposto neste TCC visa-se um sistema de baixo custo com o uso de tags passivas.

%
%
\section{\textit{Real-Time Locating Systems Using Active RFID for Internet of Things}}

O trabalho de \citeonline{realtimeRFID2016} propõe um sistema de localização em tempo real, o iLocate que 
é mais um sistema que faz uso de tags ativas, e utilizar-se de ZigBee\todo{Adicionar referência} para assim assegurar 
que a transmissão de dados a longa distância seja garantida. 

O sistema proposto por \citeonline{realtimeRFID2016} inicialmente aplica uma técnica de \textit{fingerprinting} que 
seria uma técnica equivalente ao reconhecimento de padrões abordada em \autoref{subsection:wlan}, iniciando com leituras 
breve de RSSI e implantação de tags de referências no mapa de interesse, seguindo para o armazenamento das 
leituras no banco de dados e construção de uma matriz com dados de RSSI. 

O iLocate conta com um refinamento para ter uma localização mais precisa, que funciona criando tags de referência virtuais 
que também utiliza a técnica de \textit{fingerprinting} para comparar com a reais. Outro recurso encontrado no iLocale é a 
comunicação tag-tag em que possibilita a troca de mensagens entre tags de referências que estão no mesmo intervalo de 
trabalho \cite{realtimeRFID2016}.


A utilização de uma base de dados por \citeonline{realtimeRFID2016} é um recurso que também será utilizado no sistema proposto 
neste trabalho, por mais que os dados armazenados sejam diferentes o banco de dados terá função importante para controle e monitoramento 
dos objetos que irão utilizar a tags passivas de RFID.


\section{\textit{Object localization using RFID}}

Na pesquisa de \citeonline{localization2010} é proposto um método de localização que permite estimar a posição de objetos 
com rapidez e velocidade utilizando a variação dos níveis de potência dos leitores, o método ainda faz uso de tags de 
referência para auxiliar na localização das tags\todo{Descrever se são tags ativas ou passivas} que serão rastreadas.


O método proposto por \citeonline{localization2010} foi aplicado em uma sala cuja as dimensões são $2$m x $3$m. 
A região que assemelha a um retângulo foi dividida em oito sub-regiões de tamanho iguais e denominadas de setores, 
em seguida cada setor foi dividido em quatro regiões de tamanho iguais chamados de quadrantes.

\par
Depois de fazer a divisão do ambiente é passado para inserção de tags de referências em cada quadrante, 
essas tags de referência são utilizadas inicialmente para calibrar os leitores RFID com uma relação de potência em 
contraste com distância. Por fim é utilizados algoritmos que ficam variando o nível de potência do leitor, esses algoritmos 
podem iniciar a variação da potência do menor para maior até alcançar um nível mínimo para localizar o objeto ou podendo utilizar 
uma variação da potência do maior para o menor \cite{localization2010}.

\par
O método \citeonline{localization2010} diferencia-se do método proposto no caso de buscar uma localização e posicionamento 
preciso em ambiente fechado, o que não se torna uma prioridade para o método proposto neste trabalho, sendo inicialmente suficiente 
para o sistema proposto a localização para controle de inventário.


\section{\textit{RFID localization algorithms and applications-A review}}

O artigo de \citeonline{rfid2009review} tem uma abordagem diferente dos trabalhos anteriores, não propondo nenhum método de 
localização mas mostrando as aplicações existentes, revisando algoritmos de localização indoor e potenciais da 
localização de RFID.
%
RFID pode ser empregado em várias aplicações, de acordo com \citeonline{rfid2009review} RFID é utilizado no porto de 
Cingapura para posicionar os contêiner, na saúde para localização de pacientes, suprimentos e equipamentos e também na gestão de 
material de construção permitindo monitoramento do andamento e status da construção.

Em \citeonline{rfid2009review} é realizado uma revisão dos algoritmos de localização classificando-os em dois grupos, 
o grupo que calibra a distribuição do sinal de radiofrequência e depois estima a posição do objeto e o segundo grupo que 
calculam diretamente a posição do objeto utilizando dados de RSSI. Dentro do primeiro grupo encontramos os algoritmo de 
multilateração e inferência bayesiana, já no segundo grupo estão os algoritmos de aprendizados em proximidade \cite{rfid2009review}.
%
O artigo de \citeonline{rfid2009review} serve de base para este trabalho de TCC em relação a definição de conceitos e estratégias para o 
posicionamento de RFID em sistemas de localização.
