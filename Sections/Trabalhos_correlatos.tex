\label{chapter:correlatos}
Este capítulo tem como objetivo apresentar os principais trabalhos relacionados com o sistema proposto neste trabalho, exaltando as diferenças e contribuições para o método proposto nesse trabalho. Os trabalhos em específico abordados utilizam tag RFID ou referenciam sistemas que fazem uso de tal ferramenta para localizar e identificar objetos em ambientes fechados: An indoor localization mechanism using active RFID tag \cite{mechanismRFID2006}, Real-Time Locating Systems Using Active RFID for Internet of Things \cite{realtimeRFID2016}, Object localization using RFID \cite{localization2010} e RFID localization algorithms and applications-A review \cite{rfid2009review}.

\section{An indoor localization mechanism using active RFID tag}
O trabalho de \citeauthor{mechanismRFID2006} faz uma análise do método utilizado no sistema LANDMARC e depois propõe uma melhoria para o método. O LANDMARC é um sistema de localização que faz uso de tag de refência. As tags de referência são tags ativas colocadas em pontos fixos que servem como pontos de referência para o sistema e calibração da localização, esse sistema utiliza tags ativas vizando diminuir a quantidade de leitores RFID para baratear o sistema.

%Funcionamento do landmarc
\par
Segundo \citeauthor{mechanismRFID2006} o LANDMARC funciona da seguinte maneira: primeiro os leitores detectam as tags de referências que estão no seu alcance, após isso é concedido ponderação as tags de referências de acordo com RSS, sendo valor mais alto para as mais próximas. O passo seguinte é realiza o calculo da distância euclidiana na RSS entre uma tag de rastreamento e uma tag de referência e quanto menor o resultado mais próximo da tag de referência. Por fim o sistema escolhe k tags de referências com os menores valores no resultado do calculo e obtem a localização da tag de rastreamento por meio da $(x,y) = \sum^k_{i=1}w_i(x_i,y_i), w_j=\frac{\frac{1}{E_i^2}}{\sum_{i=1}^k\frac{1}{E^2_i}}$. 
%Melhoria proposta
\par
\citeauthor{mechanismRFID2006} mostra que LANDMARC possui alguns problemas, sendo efetuando cálculos desnecessários durante a escolha das tags vizinhas e comparação de valores RSS de leitores que não estão no alcance da tag alvo. Para melhorar o sistema \citeauthor{mechanismRFID2006} propõe a utilização apenas dos leitores RFID  e tags de referência que alcançam o alvo, sendo que o númeor de leitores e das tags de referência tem que ser maior que 3, pois depois o método consiste em utilizar a técnica de Triangulação.
%Comparação com o método proposto
\par
O método utilizado por \citeauthor{mechanismRFID2006} tem objetivo de ter alcançar grande precisão no posicionamento em relação ao ambiente, já o método proposto nesse trabalho tem objetivo de saber exatamente a sala que os objetos estão, tendo em vista que o ambiente possui inúmeras salas. Entretando essa aplicação para posicinamento na sala pode ser acrescentado e estudada em trabalhos futuros.

\section{Real-Time Locating Systems Using Active RFID for Internet of Things}

\section{Object localization using RFID}

\section{RFID localization algorithms and applications-A review}