
\label{chapter:metodo}
Este Capítulo tem o objetivo de descrever a metodologia proposta neste trabalho, visando localizar e identificar objetos em ambientes confinados utilizando leitores e etiquetas passivas RFID, para gerenciamento e controle de bens.

\section{Visão geral do método}
O método consiste em um sistema que localiza qualquer objeto que tenha uma etiqueta RFID passiva fixada em seu corpo, as etiquetas serão localizadas e identificadas a medida em que transitam de uma sala para outra em um edifício. A localização e identificação é feita por meio de leitores RFID e controladores colados próximos a porta e os leitores devem ler tags que passam pela porta.
\par
Cada leitor junto ao controlador será responsável por uma sala, portanto qualquer etiqueta lida por aquele leitor ocasionará nas ações referentes a mesma sala. As ações poderá ser de atualização sobre a localização do objeto ou informando que o objeto está em transição para outra sala no edifício.
\begin{figure}[H]
              \caption{\label{fig:modelo}{Modelo Proposto}}
              \centering
              \includegraphics[width=1.1\textwidth]{Figuras/bigpicture.png}
              \legend{Fonte: Própria}
        \end{figure}
\par
Na \autoref{fig:modelo} é possivel notar que há um dispositivo acoplado proximo a porta de cada sala, esse dispositivo cotém o leitor RFID, esse dispositvo será encarrecado de consultar um servidor para assim tomar decisões do que será feito com o status do objeto, cada objtetos possui uma tag RFID fixada nele e quando passar pela porta o leitor identifica o objeto e altera sua as informações sobre sua localização.

\section{Fluxo de execução do método}
Os dispositivos de cada sala terão o seguinte fluxo de execução apresentado na \autoref{fig:fluxograma}. Esse fluxo exemplifica as tarefas que cada dispositivos executarão.
\begin{figure}[H]
              \caption{\label{fig:fluxograma}{Fluxograma de cada sala}}
              \centering
              \includegraphics[width=1\textwidth]{Figuras/fluxograma.png}
              \legend{Fonte: Própria}
        \end{figure}
\par
O fluxograma é separado em três modulos: \textbf{a}, \textbf{b} e \textbf{c}. O modulo \textbf{a} é onde é realizado o monitoramento da da sala, realizando leituras dos objetos que entram e saem da sala, no módulo \textbf{b} é o onde serão tomadas as decisões referente a etiqueta anexada ao objeto lida naquele momento, verificando se a etiqueta está ou não localizada naquela sala e processando as alterações necessárias. O módulo \textbf{c} é o último processo que atualiza as informações do servidor para que os dados consultados sejam iguais para todos.
\section{Localização dos objetos}
Os objetos serão localizado no sistema baseado nos algoritmos de proximidade, ou seja, a partir do momento em que o leitor RFID identifica uma etiqueta, esse objeto terá a localização referente à aquele leitor, nesse sistema os leitores irão representar salas, portanto a localização indicará que o objeto estará na sala cujo o leitor que realizou a leitura representa.
\par
O sistema conterá um servidor que possui a função de tratar problemas relacionados a leituras de uma mesma etiqueta mais de uma vez, o papel do servidor é de grande importância também sendo utilizado para consultar informações referentes aos objetos. 

