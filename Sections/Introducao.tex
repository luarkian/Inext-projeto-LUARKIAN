
%===========================================================
%INTRODUÇÃO
%===========================================================
A localização é um tipo de informação situacional muito importante, no contexto geral temos aplicações que fornecem a 
localização de algo, por exemplo para casas nós temos ruas, número da casa, e bairro. No sistema de posicionamento global 
para um ponto temos latitude e longitude. Entretanto quando se restringe o ambiente considerando apenas edifícios e o 
alvo para itens de uso geral (cadeiras, mesas, notebooks, desktop, etc) seja para realizar monitoramento, acompanhamento do 
fluxo e/ou rastreamento poucas ferramentas proporcionam isso com baixo custo \cite{rfid2009review}.


As conexões entre computadores que chamavam de 
redes de computadores quando se referia a um conjunto de computadores autônomos interconectados \cite{tenenbaum2002}, 
esse conceito está absoleto pois não se tem apenas computadores conectados, mas diversos dispositivos que podem ser ou 
não semelhantes, podendo ser cameras, sensores, \textit{smartphones}, ou qualquer objeto que possua um 
hardware com capacidade de conectividade \cite{iot2016SBRC}.


Com os avanços tecnológios interações com objetos tornou-se cada vez mais frequênte, isso se deu a um paradigma 
que visa uma interligação de objetos em rede para interação e cooperação podendo mudar ou não a forma como as atividades 
rotineiras serão realizadas, esse modelo é chamado de \textit{Internet of Things (IoT)} \cite{realtimeRFID2016}.


A IoT que também está relacionada com computação ubíqua é chamada de tecnologia do futuro, a comunicação de diferentes dispositivos, 
edificios ou qualquer equipamente que possua um microcontrolador ou microprocessador embutido e que podem se conectar a 
redes para assim ter uma comunicação com outros se encaixa nessa relação 
\cite{mechanismRFID2006}.


No trabalho de \citeonline{IotMartins} é realizado uma abordagem para \textit{smart cities}, que são 
aplicações que visam a troca de inormações entre veículos, \textit{smartphones}, semáforos e qualquer 
outro dispositivo capaz de enviar e receber dados, com a troca de dados entre os dispositivos pretende-se 
ter um melhor fluxo de carros em uma cidade, gerenciamento de desperdicios e monitoramento de qualidade de vida na cidade. 


\citeonline{landmarc} apresenta um sistema de localização RFID, que foi aplicado para facilitar a gestão de 
hospitais e outras organizações, para melhorar os seviços, economizar custos e reduzirem os riscos. O sistema também era 
utilizado para monitorar pacientes infcciosos e garantir resposta no tempo adequado para emergências.


A interligação dos objetos as redes possibilitou que o meio científico propursessem várias maneiras de localização em ambientes 
confinados, visto que o GPS não funciona tão bem para a solução proposta, tendo em vista que um edifício possui 
muitas salas ou/e andares, e com o GPS não seria possível saber em qual sala por não ter uma planta baixa do prédio ou 
informar qual o andar o objeto se encontra pois o GPS retornaria para a aplicação as coordenadas de latitude e longitude \cite{rfid2009review}.


Para solucionar essa incapacidade de localização do GPS, inúmeas aplicações foram propostas para a localização indoor. 
Aplicações que utilizam tecnologia infravermelho difuso para estimar a posição, sistemas que utilizam adaptadores de rede 
padrão IEEE 802.11 para rastrear os objetos no interior do prédio, sistemas baseados na tecnologia ultra-sônica e 
tambem sistemas que utilizam RFID para rastreio indoor foram propostos \cite{mechanismRFID2006}. 


O RFID foi a tecnologia escolhida por proporcinar a utilização de etiquetas que não necessitam de uma fonte de alimentação 
para seu funcionamento, pois são alimentadas pelas ondas eletromagnéticas emitidas pela antena do leitor, sendo que isso não 
é possivel em outras tecnologias. Uma outra funcionalidade do RFID é a possibilidade de identificação das etiquetas que possui 
uma ID única.


% %===========================================================
% %MOTIVAÇÃO
% %===========================================================

%===========================================================
%DEFINIÇÃO DO PROBLEMA
%===========================================================
\section{Definição do Problema}
A dificuldade de localizar objetos em ambientes confinados ou fechados, cria uma necessidade de novas ferramentas para 
tal aplicação sendo que o GPS não se aplica a tal problema por na localização do GPS os prédios são apresentados como um edifício sem salas e divisórias, sem contar o custo beneficio para anexar módulos de GPS aos objetos, isso iria deixar o protótipo caro. Contudo, o motivo desta pesquisa é auxiliar na localização de objetos permanentes (como computadores e moveis) 
em ambientes de uma forma que não tenha o custo elevado quanto a utiliação do 
GPS\cite{mechanismRFID2006}.

\par
\begin{comment}
Os edifícios e construções dificultam o envio de sinais de rádio emitidos pelos satélites e dispositivos que fazem uso do GPS, 
por essa questão sua utilização torna-se inviável para aplicações que consistem em localizar objetos em ambientes fechados, 
além que o tempo-de-luz transitório fica difícil e caro para fazer sua medição \cite{rfid2009review}.
\end{comment}

%dependendo da forma que os dados são armazedados e recuperados possibilita a identificação automatica utilizando tags e leitores, 

Localizar e identificar objetos em certos ambientes têm grande importância quando se quer gerenciar e controlar ativos, 
por exemplo em comércios, indústrias ou instituições de ensino que possuem uma gama de bens que podem ser movimentados dentro do local, 
poder identificar o objeto que está sendo movido e saber em que local o objeto está, é de grande importância para ter um maior 
controle sobre os bens \cite{realtimeRFID2016}. 

A tecnologia RFID é um meio de comunicação sem fio que utiliza compo eletromagnéticos de 
radiofrequencia para identificar etiquetas. As etiquetas podem ser passivas, ativas e 
semi-passivas. As etiquetas semi-ativa e ativas possuem uma fonte de alimentação, a semi-ativa só emite sinais 
quando entra em contato com o leitor, já ativa emite os sinais a todo momento, as etiquetas passivas não 
possuem fonte de alimentação entrando em funcionanmento apenas quando as ondas eletromagnéticas emitidas pelo 
leitor os alimentam\cite{realtimeRFID2016}. As funcionalidade da tecnoliga RFID fez com que muitas técnicas de localização fossem propostas.

\begin{comment}
\par
Muitas técnicas de localização baseadas em RFID foram propostas, sendo com foco em objetos móveis ou estacionários, contudo algumas dessas 
técnicas fazem uso de etiquetas ativas a fim de obter melhores estimativas,
\end{comment}


O problema considerado neste trabalho é expresso na seguinte questão: 
\textbf{Como projetar um sistema computacional capaz de localizar e identificar objetos em tempo real com baixo custo, em ambientes confinados ou fechados, garantindo confiabilidade na informações fornecidade e auxiliando usuários no gerenciamento de bens por meio da utilização de radio frequência?}

%===========================================================
%OBJETIVOS GERAIS E ESPECIFICOS
%===========================================================
\section{Objetivos}
O objetivo principal deste trabalho é projetar e desenvolver um sistema computacional autonomo para o gerenciamento de objetos com RFID 
em ambientes confinados ou fechados, utilizando técnicas de localização e identificação de redes IoT.


Objetivos específicos:
\begin{enumerate}

    \item Analisar soluções computacionais de baixo custo que são capazes de localizar objetos indoor;
    
    \item Identificar algoritmos para localização de objetos indoor;
        %, verificando se podem ser adaptados no prótotipo a ser desenvolvido;
%    \item Localizar objetos em ambientes confinados utilizando tags RFID.
    
%    \item Identificar objetos em ambientes internos utilizando tags RFID.
    
%    \item Monitorar objetos caso mudem de localização no ambiente, a fim de ter um controle sobre os objetos cadastrados no sistema.
    
    \item Projetar e desenvolver um prótotipo de um sistema capaz de localizar e identificar objetos em tempo real, no âmbito confinado utilizando RFID; e
    
    \item Validar o método proposto com o próposito de examinar sua eficácia e aplicabilidade.
\end{enumerate}


\begin{comment}
%===========================================================
%METODOLOGIA PROPOSTA
%===========================================================
\section{Metodologia Proposta}

%===========================================================
%CONTRIBUIÇÕES PROPOSTAS
%===========================================================
\section{Contribuições propostas}
As contribuições propostas deste trabalho são:
\begin{enumerate}
    \item A implementação de um sistema para localização de objetos. O Metódo utilizado visa localizar objeto sem alta precisão, porém é viável para controle de acervos.
    \item O sistema desevolvido pode auxiliar no controle e ainda facilitar o levantamento de todos os bens do proprietário.
\end{enumerate}

\end{comment}


%===========================================================
%ORGANIZAÇÃO DO TRABALHO
%===========================================================
\section{Organização do trabalho}
A introdução deste trabalho apresentou: o contexto, definição do problema, objetivos, metodologia e contribuições 
dessa pesquisa. Os capítulos restantes são organizados da seguinte forma:

No \autoref{chapter:conceitos}, \textbf{Conceitos e Definições}, são apresentados fundamentos teóricos que abordam os 
seguintes assuntos: sistemas embarcados, sistemas de comunicação e localização, algoritmos utilizados em localização, e modelagem.

\par
No \autoref{chapter:correlatos}, \textbf{Trabalhos Correlatos}, são apresentados trabalhos correlatos a utilização de RFID e localização.

\par
No \autoref{chapter:metodo}, \textbf{Método Proposto}, é descrito as etapas da solução proposta neste trabalho, de 
forma a mostrar a arquitetura do sistema proposto e componentes necessários para a aplicação.

\par
No \autoref{chapter:cronograma}, \textbf{Cronograma}, é descrito as próximas fases do desenvolvimento deste trabalho.

%No \autoref{chapter:resultados} \textbf{Resultados Experimentais}, descreve-se a execução de uma avaliação experimental 
\par
E por fim no \autoref{chapter:consideracoes}, \textbf{Considerações Parciais e Trabalhos Futuros}, será apresentado as 
considerações parciais e próximos passo deste trabalho.
