
%===========================================================
%INTRODUÇÃO
%===========================================================
As conexões entre computadores que chamavam de "redes de computadores" quando se referia a um conjunto de computadores autônomos interconectados \cite{tenenbaum2002}, esse conceito está absoleto pois não se tem apenas computadores conectados, mas diversos dispositivos que podem ser ou não semelhantes, podendo ser cameras, sensores, smartphones, ou qualquer objeto que possua um hardware com capacidade de conectividade \cite{iot2016SBRC}.
\par
Com os avanços tecnológios interações com objetos tornou-se cada vez mais frequênte, isso se deu a um paradigma que visa uma interligação de objetos em rede para interação e cooperação podendo mudar ou não a forma como as atividades rotineiras serão realizadas, esse modelo é chamado de \textit{Internet of Things (IoT)} \cite{realtimeRFID2016}.
\par
A IoT que também está relacionada com computação ubíqua é chamada de tecnologia do futuro, a comunicação de diferentes dispositivos, edificios ou qualquer equipamente que possua um microcontrolador ou microprocessador embutido e que podem se conectar a redes para assim ter uma comunicação com outros se encaixa nessa relação \cite{mechanismRFID2006}.
\par
 A interligação dos objetos as redes possibilitou que o meio cientifico propursessem várias maneiras de localização em ambientes confinados, visto que o GPS não funciona tão bem em tais ambientes \cite{mechanismRFID2006}, e a utilização de leitores e tags RFID ganhou grande foco já que além de localizar possibilita a identiicação.
% %===========================================================
% %MOTIVAÇÃO
% %===========================================================
 \section{Motivação}
A dificuldade de localizar objetos em ambientes confinados ou fechados, cria uma necessidade de novas ferramentas para tal aplicação sendo que o GPS não se aplica a tal problema ou encontra dificuldade para funcionar de maneira correta. Contudo o motivo desta pesquisa é trabalhar uma ferramenta que possa auxiliar ou ter o papel de localizar objetos em ambientes em que não é possível fazer uso de GPS, ou seja, localizar objetos em ambientes fechados ou confinados \cite{mechanismRFID2006}.

%Porque GPS não se aplica em ambientes fechados
\par
Os edifícios e construções dificultam o envio de sinais de rádio emitidos pelos satélites e dispositivos que fazem uso do GPS, por essa questão sua utilização torna-se inviável para aplicações que consistem em localizar objetos em ambientes fechados, além que o tempo-de-luz transitório fica difícil e caro para fazer sua medição \cite{rfid2009review}.

%===========================================================
%DEFINIÇÃO DO PROBLEMA
%===========================================================
\section{Definição do Problema}

Localizar e identificar objetos em certos ambientes têm grande importância quando se quer gerenciar e controlar ativos, por exemplo em comércios, indústrias ou instituições de ensino que possuem uma gama de bens que podem ser movimentados dentro do local,  poder identificar o objeto que está sendo movido e saber em que local o objeto está, é de grande importância para ter um maior controle sobre os bens \cite{realtimeRFID2016}. 

\par
O problema considerado neste trabalho é expresso na seguinte questão: \textbf{ Como utilizar RFID para identificar e localizar bens em ambientes confinados ou fechados, para gerenciar e controlar em tempo real?}
%===========================================================
%OBJETIVOS GERAIS E ESPECIFICOS
%===========================================================
\section{Objetivos}
O objetivo principal deste trabalho é analisar, se possível utilizar algumas das técnicas existentes de localização indoor para localizar e identificar objetos em ambientes internos por meio de tags RFID.
\par
Objetivos específicos:
\begin{enumerate}

    \item Localizar objetos em ambientes confinados utilizando tags RFID.
    
    \item Identificar objetos em ambientes internos utilizando tags RFID.
    
    \item Monitorar objetos caso mudem de localização no ambiente, a fim de ter um controle sob os objetos cadastrados no sistema.
    
\end{enumerate}



%===========================================================
%METODOLOGIA PROPOSTA
%===========================================================
\section{Metodologia Proposta}

%===========================================================
%CONTRIBUIÇÕES PROPOSTAS
%===========================================================
\section{Contribuições propostas}
As contribuições propostas deste trabalho são:
\begin{enumerate}
    \item A implementação de um sistema para localização de objetos. O Metódo utilizado visa localizar objeto sem alta precisão, porém é viável para controle de acervos.
    \item O sistema desevolvido pode auxiliar no controle e ainda facilitar o levantamento de todos os bens do proprietário.
\end{enumerate}


%===========================================================
%ORGANIZAÇÃO DO TRABALHO
%===========================================================
\section{Organização do trabalho}
A introdução deste trabalho apresentou: o contexto, definição do problema, objetivos, metodologia e contribuições dessa pesquisa. Os capítulos restantes são organizados da seguinte forma:

\par
No \autoref{chapter:conceitos} \textbf{Conceitos e Definições}, são apresentados fundamentos teóricos que abordam os seguintes assuntos: sistemas embarcados, localização de objetos, algoritmos utilizados em localização e modelagem de sistemas.

\par
No \autoref{chapter:correlatos} \textbf{Trabalhos Correlatos}, são apresentados trabalhos correlatos a utilização de RFID e localização.

\par
No \autoref{chapter:metodo} \textbf{Método Proposto}, é descrito as etapas de desenvolvimento deste trabalho, de forma a mostrar a arquitetura do sistema proposto e componentes necessários para a aplicação.

\par
%No \autoref{chapter:resultados} \textbf{Resultados Experimentais}, descreve-se a execução de uma avaliação experimental 
\par
E por fim no \autoref{chapter:consideracoes} \textbf{Considerações parciais e trabalhos futuros}, será apresentado as considerações parciais e trabalhos futuros.
