
%===========================================================
%INTRODUÇÃO
%===========================================================
As\todo{Adicionar um texto motivador para abordar o assunto} conexões entre computadores que chamavam de 
"redes de computadores" quando se referia a um conjunto de computadores autônomos interconectados \cite{tenenbaum2002}, 
esse conceito está absoleto pois não se tem apenas computadores conectados, mas diversos dispositivos que podem ser ou 
não semelhantes, podendo ser cameras, sensores, \textit{smartphones}, ou qualquer objeto que possua um 
hardware com capacidade de conectividade \cite{iot2016SBRC}.


Com os avanços tecnológios interações com objetos tornou-se cada vez mais frequênte, isso se deu a um paradigma 
que visa uma interligação de objetos em rede para interação e cooperação podendo mudar ou não a forma como as atividades 
rotineiras serão realizadas, esse modelo é chamado de \textit{Internet of Things (IoT)} \cite{realtimeRFID2016}.


A IoT que também está relacionada com computação ubíqua é chamada de tecnologia do futuro, a comunicação de diferentes dispositivos, 
edificios ou qualquer equipamente que possua um microcontrolador ou microprocessador embutido e que podem se conectar a 
redes para assim ter uma comunicação com outros se encaixa nessa relação \cite{mechanismRFID2006}\todo{Apresentar um exemplo prático e real, tipo smart city e outros}.


A interligação dos objetos as redes possibilitou que o meio cientifico propursessem várias maneiras de localização em ambientes 
confinados, visto que o GPS não funciona tão bem em tais ambientes \cite{mechanismRFID2006}, e a utilização de leitores e tags RFID 
ganhou grande foco já que além de localizar possibilita a identificação\todo{Descrever outras soluções além do RFID e pq a escolha do RFID}.


% %===========================================================
% %MOTIVAÇÃO
% %===========================================================
 \section{Motivação}
A dificuldade de localizar objetos em ambientes confinados ou fechados, cria uma necessidade de novas ferramentas para 
tal aplicação sendo que o GPS não se aplica a tal problema ou encontra dificuldade para funcionar de maneira correta, exemplo, 
localização de carros em tuneis. Contudo, o motivo desta pesquisa é auxiliar na localização de objetos permanentes (como computadores e moveis) 
em ambientes em que não é possível fazer uso de GPS\todo{Tem melhorar isso, visto que o GPS ainda funciona dentro de um predio, no nosso caso nos blocos}, 
ou seja, localizar objetos em ambientes fechados ou confinados \cite{mechanismRFID2006}.

%Porque GPS não se aplica em ambientes fechados
\par
Os edifícios e construções dificultam o envio de sinais de rádio emitidos pelos satélites e dispositivos que fazem uso do GPS, 
por essa questão sua utilização torna-se inviável para aplicações que consistem em localizar objetos em ambientes fechados, 
além que o tempo-de-luz transitório fica difícil e caro para fazer sua medição \cite{rfid2009review}\todo{Falta falar qual o papel do RFID.}.

%===========================================================
%DEFINIÇÃO DO PROBLEMA
%===========================================================
\section{Definição do Problema}

Localizar e identificar objetos em certos ambientes têm grande importância quando se quer gerenciar e controlar ativos, 
por exemplo em comércios, indústrias ou instituições de ensino que possuem uma gama de bens que podem ser movimentados dentro do local, 
poder identificar o objeto que está sendo movido e saber em que local o objeto está, é de grande importância para ter um maior 
controle sobre os bens \cite{realtimeRFID2016}. 

\par
O problema considerado neste trabalho é expresso na seguinte questão\todo{Ainda podemos mehorar}: 
\textbf{Como projetar um sistema computacional capaz de localizar e identificar objetos em tempo real, em ambientes confinados ou fechados,  
por meio da utilização de radio frequência?}

%===========================================================
%OBJETIVOS GERAIS E ESPECIFICOS
%===========================================================
\section{Objetivos}
O objetivo principal deste trabalho é projetar e desenvolver um sistema computacional autonomo para o gerenciamento de objetos com RFID 
em ambientes confinados ou fechados, utilizando técnicas de localização e identificação de redes IoT.


Objetivos específicos\todo{Estes objetivos nao estão bons, tem que melhorar}:
\begin{enumerate}

    \item Localizar objetos em ambientes confinados utilizando tags RFID.
    
    \item Identificar objetos em ambientes internos utilizando tags RFID.
    
    \item Monitorar objetos caso mudem de localização no ambiente, a fim de ter um controle sobre os objetos cadastrados no sistema.
    
\end{enumerate}


\begin{comment}
%===========================================================
%METODOLOGIA PROPOSTA
%===========================================================
\section{Metodologia Proposta}

%===========================================================
%CONTRIBUIÇÕES PROPOSTAS
%===========================================================
\section{Contribuições propostas}
As contribuições propostas deste trabalho são:
\begin{enumerate}
    \item A implementação de um sistema para localização de objetos. O Metódo utilizado visa localizar objeto sem alta precisão, porém é viável para controle de acervos.
    \item O sistema desevolvido pode auxiliar no controle e ainda facilitar o levantamento de todos os bens do proprietário.
\end{enumerate}

\end{comment}
%===========================================================
%ORGANIZAÇÃO DO TRABALHO
%===========================================================
\section{Organização do trabalho}
A introdução deste trabalho apresentou: o contexto, definição do problema, objetivos, metodologia e contribuições dessa pesquisa. Os capítulos restantes são organizados da seguinte forma:

\par
No \autoref{chapter:conceitos} \textbf{Conceitos e Definições}, são apresentados fundamentos teóricos que abordam os seguintes assuntos: sistemas embarcados, localização de objetos, algoritmos utilizados em localização, modelagem e prototipação de sistemas.

\par
No \autoref{chapter:correlatos} \textbf{Trabalhos Correlatos}, são apresentados trabalhos correlatos a utilização de RFID e localização.

\par
No \autoref{chapter:metodo} \textbf{Método Proposto}, é descrito as etapas de desenvolvimento deste trabalho, de forma a mostrar a arquitetura do sistema proposto e componentes necessários para a aplicação.

\par
%No \autoref{chapter:resultados} \textbf{Resultados Experimentais}, descreve-se a execução de uma avaliação experimental 
\par
E\todo{Ajustar a númeração e seria bom ter um seção com resultados parciais apresentado um exemplo com RFID} por fim no \autoref{chapter:consideracoes} \textbf{Considerações parciais e trabalhos futuros}, será apresentado as considerações parciais e trabalhos futuros.
