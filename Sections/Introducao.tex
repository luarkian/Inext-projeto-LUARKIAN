
%===========================================================
%INTRODUÇÃO
%===========================================================

% %===========================================================
% %MOTIVAÇÃO
% %===========================================================
 \section{Motivação}
A dificuldade de localizar objetos em ambientes confinados ou fechados, cria uma necessidade de novas ferramentas para tal aplicação sendo que o GPS não se aplica a tal problema ou encontra dificuldade para funcionar de maneira correta. Contudo o motivo desta pesquisa é trabalhar uma ferramenta que possa auxiliar ou ter o papel de localizar objetos em ambientes em que não é possível fazer uso de GPS, ou seja, localizar objetos em ambientes fechados ou confinados \cite{mechanismRFID2006}.

%Porque GPS não se aplica em ambientes fechados
\par
Os edifícios e construções dificultam o envio de sinais de rádio emitidos pelos satélites e dispositivos que fazem uso do GPS, por essa questão sua utilização torna-se inviável para aplicações que consistem em localizar objetos em ambientes fechados, além que o tempo-de-luz transitório fica difícil e caro para fazer sua medição \cite{rfid2009review}.

%===========================================================
%DEFINIÇÃO DO PROBLEMA
%===========================================================
\section{Definição do Problema}

Localizar e identificar objetos em certos ambientes têm grande importância quando se quer gerenciar e controlar ativos, por exemplo em comércios, indústrias ou instituições de ensino que possuem uma gama de bens que podem ser movimentados dentro do local,  poder identificar o objeto que está sendo movido e saber em que local o objeto está, é de grande importância para ter um maior controle sobre os bens \cite{realtimeRFID2016}. 

\par
O problema considerado neste trabalho é expresso na seguinte questão: \textbf{ Como utilizar RFID para identificar e localizar bens em ambientes confinados ou fechados, para gerenciar e controlar em tempo real?}
%===========================================================
%OBJETIVOS GERAIS E ESPECIFICOS
%===========================================================
\section{Objetivos}
O objetivo principal deste trabalho é analisar, se possível melhorar as técnicas existentes de localização utilizando RFID em ambientes internos e propor uma forma de localizar e identificar objetos em ambientes internos por meio de RFID.
\par
Objetivos específicos:
\begin{enumerate}

    \item Localizar e identificar objetos em ambientes internos utilizando RFID.
    
    \item Obter o posicionamento dos objetos para uma localização mais precisa.
    
    \item Monitorar objetos caso mudem de posição ou localização no ambiente, a fim de ter um controle sob os objetos cadastrados no sistema.
    
\end{enumerate}



%===========================================================
%METODOLOGIA PROPOSTA
%===========================================================
\section{Metodologia Proposta}


\begin{comment}
%===========================================================
%CONTRIBUIÇÕES PROPOSTAS
%===========================================================
\section{Contribuições propostas}



%===========================================================
%ORGANIZAÇÃO DO TRABALHO
%===========================================================
\section{Organização do trabalho}
A introdução deste trabalho apresentou: o contexto, definição do problema, objetivos, metodologia e contribuições dessa pesquisa. Os capítulos restantes são organizados da seguinte forma:

\par
No \autoref{chapter:conceitos} \textbf{Conceitos e Definições}, são apresentados os conceitos abordados neste trabalho, especificamente: Linguagens de descrição de hardware, Verificação e Validação de Sistemas e Técnicas de Compiladores.

\par
No \autoref{chapter:correlatos} \textbf{Trabalhos Correlatos}, será apresentado o método de pesquisa utilizado, mas também, os resultados alcançados com a pesquisa, análise dos artigos e a contribuição dos mesmos para o desenvolvimento do método apresentado neste trabalho.

\par
No \autoref{chapter:metodo} \textbf{Método Proposto}, é descrito as etapas de execução do novo método proposto. Em especial, são descritos o método de transformação do código, a utilização das assertivas e a integração da ferramenta ESBMC no contexto do método.

\par
No \autoref{chapter:resultados} \textbf{Resultados Experimentais}, descreve-se a execução de uma avaliação experimental sobre o método proposto, bem como, 
\textit{benchmarks} utilizados para testes da ferramenta e os resultados obtidos através destes testes.
\par
E por fim no \autoref{chapter:consideracoes} \textbf{Considerações parciais e trabalhos futuros}, apresenta-se as considerações parciais e os trabalhos futuros a serem desenvolvidos. 

\end{comment}