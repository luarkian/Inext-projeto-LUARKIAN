
\label{chapter:consideracoes}

Este trabalho propôs um sistema de localização e identificação de objetos em edifícios por meio de
radiofrequência, usando etiquetas passivas. O real objetivo consiste em facilitar e auxiliar o gerenciamento/controle de
bens em edifícios cujo possuem uma grande quantidade de salas.
\par
O método até o momento possui limitações referente ao leitor RFID utilizado, o leitor possui o alcance baixo necessitando que as etiquetas sejam praticamente encostadas para que a leitura seja realizada, sendo necessário a substituição do leitor caso o sistema seja implantado em um prédio.
\par

%Neste trabalho já foram obtidos alguns resultados parciais em relação a identificação de componentes de hardware como o módulo wireless ESP-8266 e o Módulo Leitor Rfid Mfrc522 Mifare, bem como, alguns testes iniciais que apontam a viabilidade para a construção de um protótipo. Desta forma, os próximos passos deste trabalho visam o desenvolvimento e testes do sistema proposto para análise da localização dos objetos. 
Adicionalmente, visa-se estudar métodos para ter uma localização mais precisa sobre a localização dos objetos, por exemplo, pode-se utilizar etiquetas ativas para auxiliar e em seguida aplicar algoritmos de RSSI, dessa forma tende-se a ter uma localização com maior precisão no ambiente e uma maneira para executar o processo de identificação dos objetos de maneira automática. 



