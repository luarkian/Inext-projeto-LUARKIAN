\section{Revisão Sistemática}

Nesta seção é apresentado o protocolo definido para a aplicação e execução da revisão sistemática que foi parcialmente executada
neste trabalho, devido ao tempo e quantidade de publicações\todo{Adicionar o número de artigos da execução da string de busca} 
identificadas. Todos os artigos citados na Seção~\ref{chapter:correlatos} foram selecionados através da execução deste protocolo proposto.
%execução de uma revisão sistemática da literatura, parcialmente executada, conforme descrito  detalhadamente no \autoref{chapter:apendice1} - Apêndice $1$.

			Segundo \citeauthor{Kitchenham:2009:SLR:1465742.1466091}  a revisão sistemática é uma seleção da literatura que segue um protocolo formal, sendo uma revisão metodologicamente rigorosa dos resultados e com o objetivo de agregar todas as evidências existentes em uma pesquisa. A sua aplicação consiste em umas sequência de passos definidos a ser seguidos e se comparado a revisão da literatura informal requer um grande esforço.
			
			\par 
			A revisão sistemática atua como um meio de identificação  e avaliação para pesquisa conduzindo-o para tópicos relevantes sobre o tema de interesse, diferentemente da revisãos da literatura informal em que a presença de variações entre os estudos representa um fator negativo
			\cite{MafraTravassos}. 
			De acordo com \citeauthor{MafraTravassos} o processo da revisão sistemática envolve três etapa: 
			\begin{enumerate}
			    \item \textbf{Planejamento da Revisão}: nesta etapa os objetivos são listado e é definido o protocolo de revisão;
			    \item \textbf{Condução da Revisão}: etapa onde as fontes para a revisão sistemática são selecionados, identificação de estudos primários e avaliados com critérios de inclusão e exclusão;
			    \item \textbf{Publicação dos Resultado}: na terceira etapa os dados dos estudos são extraídos e sintetizados para serem publicados.
			\end{enumerate}
			
			\par 
			Conduzimos a revisão deste trabalho com a finalidade de realizar um estudo exploratório de caracterização da área, e assim podemos afirmar que essa revisão se caracteriza como uma quasi-sistemática pois está baseada nas três etapas citada anteriormente \cite{MafraTravassos}, seguindo o processo da revisão sistemática e preservando o rigor e o mesmo formalismo para as fases metodológicas de elaboração e execução do protocolo, faltando apenas uma meta-análise do que poderia ser aplicado futuramente.
			

		\subsection{ Planejamento da Revisão Sistemática}
			\par
\textbf{Objetivo: }O objetivo desta revisão é analisar publicações científicas com o propósito de  identificar métodos para a localização de objetos com a utilização de tag RFID no contexto acadêmico ou industrial.
			
            \par
            \textbf{Formulação das Perguntas.}
            
            \begin{itemize}
              \item Quais os métodos para a localização de tags RFID?
              \item Está disponível alguma ferramenta para localização de tags RFID ?
              \item Quais as limitações em relação a identificação de mais de uma tag RFID simultaneamente?
              \item Quais os requisitos necessários para execução do método?
              \item Quais as limitações do método proposto ?
              \item Quais são as perspectivas futura para a melhoria do método e da aplicação?

            \end{itemize}
			
            \begin{itemize}
            \item \textbf{População}
            	\begin{itemize}
            		\item \textbf{Palavras-Chave} "tag RFID"  OR  "RFID signaling scheme"  OR  "RFID reader"  OR  "Handheld RFID Reader"  OR  "RFID Label"  OR  "RFID Tag Reader"  OR  "Passive RFID Tags"  OR  "Passive RFID Readers" OR “Radio-Frequency IDentification” OR “Radio Frequency IDentification”  OR “review of RFID” OR “RFID  technology” OR “RFID localization” OR “method for RFID” OR “Active RFID”
            	\end{itemize}
            \item \textbf{Intervenção } 
            	\begin{itemize}
            
             \item \textbf{Palavras-Chave} “Positioning principles” OR "object localization indoor"  OR  "positioning algorithms"  OR  "indoor location technique"  OR  "indoor object management"  OR  "asset localization"  OR  "localization-based services"  OR  "facilities management"  OR  "pattern matching"  OR  "mapping"  OR  "indoor automatic identification"  OR  "indoor tag classification"  OR  "indoor classify RFID"  OR  "indoor classify tag"  OR  "indoor localization techniques" OR “Applications and techniques” OR “system for indoor” OR “real-time locating” OR “localization method” OR “tags for indoor” OR “Efficient object localization” OR “using RSS” OR “algorithms and applications” AND NOT “using UHF RFID”
           		\end{itemize}
            \end{itemize}
             
            
	\subsection{Procedimento de Seleção de Critérios}

	  		\begin{itemize}
      			\item \textbf{CE1-01}: Não serão selecionadas publicações em que não contém as  palavras-chave da string de busca no título, resumo, método  e/ou resultados. Os campos de seções de agradecimentos, biografia dos autores, referências bibliográficas e anexos não serão incluídos.

      			\item \textbf{CE1-02}: Publicações cujo se assemelham com  tutoriais e cursos não serão selecionadas.

      			\item \textbf{CE1-03}: Não serão selecionadas publicações em que técnicas e algoritmos  para localização de objetos não trabalham com identificação por rádio frequência (RFID).

      			\item \textbf{CE1-04}: Não serão selecionadas publicações em que façam utilização de frequência ultra-alta (UHF) RFID.

      			\item \textbf{CE1-05}:  Não serão selecionadas publicações que não fazem uma abordagem para utilização de RFID para localizar objetos com tags RFID.
    	\end{itemize}
	\par
	Publicações que podem ser incluídas no conjunto.

	\begin{itemize}
    	\item \textbf{CI1-01}: Podem ser selecionadas publicações em que o contexto das palavras-chave utilizadas no artigo leva a crer que faz uma citação para uma abordagem de utilização de RFID para localizar objetos utilizando tags RFID.
        \item \textbf{CI1-02} Podem ser selecionadas publicações em que o contexto das palavras-chave utilizadas no artigo leva a crer que a publicação cita recomendações de algoritmos, métodos e ferramentas que possam ser utilizados para localizar objetos com tag RFID.
    \end{itemize}
