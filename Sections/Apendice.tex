\section{Revisão Sistemática}
		\subsection{ Planejamento da Revisão Sistemática}
			\par
\textbf{Objetivo: }O objetivo desta revisão é analisar publicações científicas com o propósito de  identificar métodos para a localização de objetos com a utilização de tag RFID no contexto acadêmico ou industrial.
			
            \par
            \textbf{Formulação das Perguntas.}
            
            \begin{itemize}
              \item Quais os métodos para a localização de tags RFID?
              \item Está disponível alguma ferramenta para localização de tags RFID ?
              \item Quais as limitações em relação a identificação de mais de uma tag RFID simultaneamente?
              \item Quais os requisitos necessários para execução do método?
              \item Quais as limitações do método proposto ?
              \item Quais são as perspectivas futura para a melhoria do método e da aplicação?

            \end{itemize}
			
            \begin{itemize}
            \item \textbf{População}
            	\begin{itemize}
            		\item \textbf{Palavras-Chave} "tag RFID"  OR  "RFID signaling scheme"  OR  "RFID reader"  OR  "Handheld RFID Reader"  OR  "RFID Label"  OR  "RFID Tag Reader"  OR  "Passive RFID Tags"  OR  "Passive RFID Readers" OR “Radio-Frequency IDentification” OR “Radio Frequency IDentification”  OR “review of RFID” OR “RFID  technology” OR “RFID localization” OR “method for RFID” OR “Active RFID”
            	\end{itemize}
            \item \textbf{Intervenção } 
            	\begin{itemize}
            
             \item \textbf{Palavras-Chave} “Positioning principles” OR "object localization indoor"  OR  "positioning algorithms"  OR  "indoor location technique"  OR  "indoor object management"  OR  "asset localization"  OR  "localization-based services"  OR  "facilities management"  OR  "pattern matching"  OR  "mapping"  OR  "indoor automatic identification"  OR  "indoor tag classification"  OR  "indoor classify RFID"  OR  "indoor classify tag"  OR  "indoor localization techniques" OR “Applications and techniques” OR “system for indoor” OR “real-time locating” OR “localization method” OR “tags for indoor” OR “Efficient object localization” OR “using RSS” OR “algorithms and applications” AND NOT “using UHF RFID”
           		\end{itemize}
            \end{itemize}
             
            
	\subsection{Procedimento de Seleção de Critérios}

	  		\begin{itemize}
      			\item \textbf{CE1-01}: Não serão selecionadas publicações em que não contém as  palavras-chave da string de busca no título, resumo, método  e/ou resultados. Os campos de seções de agradecimentos, biografia dos autores, referências bibliográficas e anexos não serão incluídos.

      			\item \textbf{CE1-02}: Publicações cujo se assemelham com  tutoriais e cursos não serão selecionadas.

      			\item \textbf{CE1-03}: Não serão selecionadas publicações em que técnicas e algoritmos  para localização de objetos não trabalham com identificação por rádio frequência (RFID).

      			\item \textbf{CE1-04}: Não serão selecionadas publicações em que façam utilização de frequência ultra-alta (UHF) RFID.

      			\item \textbf{CE1-05}:  Não serão selecionadas publicações que não fazem uma abordagem para utilização de RFID para localizar objetos com tags RFID.
    	\end{itemize}
	\par
	Publicações que podem ser incluídas no conjunto.

	\begin{itemize}
    	\item \textbf{CI1-01}: Podem ser selecionadas publicações em que o contexto das palavras-chave utilizadas no artigo leva a crer que faz uma citação para uma abordagem de utilização de RFID para localizar objetos utilizando tags RFID.
        \item \textbf{CI1-02} Podem ser selecionadas publicações em que o contexto das palavras-chave utilizadas no artigo leva a crer que a publicação cita recomendações de algoritmos, métodos e ferramentas que possam ser utilizados para localizar objetos com tag RFID.
    \end{itemize}