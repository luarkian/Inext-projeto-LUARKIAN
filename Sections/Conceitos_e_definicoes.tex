\label{chapter:conceitos}
O principal objetivo desse capítulo é apresentar conceitos necessários para o entendimento deste trabalho, de forma clara e direta. Os assuntos abordados neste capítulo são: Comunicação sem fio, Localização e Algoritmos de localização RFID.
\section{Sistemas Embarcados}
    \par
    Segundo \citeauthor{rodrigo2016} sistemas embarcados estão presente em quase todos os ambientes, tais sistemas possuem uma única função específica e que não pode ser alterada. Eles são controlados por microprocessadores ou microcontroladores de forma que possuem muitas restrições em relação a recursos computacionais.

    \par
    Atualmente é possível encontrar sistemas embarcados em diversos dispositivos, por exemplo: televisores, micro-ondas, sistemas de gerenciamento de aviação, esteiras, etc. Os dispositivos que fazem uso de eletricidade para seu funcionamento, basicamente possuem um sistema embarcado para articular o seu funcionamento \cite{rodrigo2016}.
    
    \par
    Na \autoref{fig:sistemas embarcados} estão alguns dos aparelhos em que é possível se encontrar sistemas embarcado. É possível notar que aparelhos antigos já utilizavam tal sistemas e com o passar dos anos cada vez mais estão sendo introduzidos nos eletrônicos.
    \begin{figure}[h!]
              \caption{\label{fig:sistemas embarcados}{Sistemas Embarcados.}}
              \centering
              \includegraphics[width=0.5\textwidth]{Figuras/systems_embedded.PNG}
              \legend{Fonte: \cite{Li:2003:RCE:829584}}
            \end{figure}
    \par
    Uma definição geral para sistemas embarcados: são sistemas que realizam uma função dedica e possuem a integração de hardware e software fortemente acoplados, geralmente são uma parte específica de um sistema maior \cite{Li:2003:RCE:829584}.
    % microcontroladores e microprocessadores
    
    % Sistemas de tempo real e tolerancia a falha
    

    \subsection{IoT}
    \par
    A possibilidade de comunicação entre objetos de uso cotidiano do ambiente real com a internet referência o termo IoT, quando um objeto está conectado a rede de computadores e passa a transmitir informações de seu funcionamento ou estado, tal objeto passa a ser denominado de objeto inteligente \cite{iot2016}.
    \par
    De acordo com \citeauthor{iot2017}, o conceito de IoT não é novo, pois desde os passos iniciais da internet ja se pensava em formas de traçar a comunicação entre objetos do dia a dia com a internet. Com os avanços de sistemas embarcados o desenvolvimento de uma infinidades de padrões e protocolos para a integração de WSN tornaram IoT uma realidade.
    
    % relação de iot com sistemas embarcados
    
\section{Comunicação Sem Fio}
    \par
    A medida que os elétrons se movimentam ondas eletromagnéticas são criada no espaço, essas ondas são medidas de acordo com suas oscilações e chamadas de frequência (Hertz), o comprimento dessa onda é medido pela distância de dois pontos máximos ou dois pontos minímos seguidos \cite{tenenbaum2002}.
    \par
    Segundo \citeauthor{tenenbaum2002} ao colocar uma antena em um circuito elétrico apropriado pode-se  transmistir e receber ondas eletromagnéticas, a comunição sem fio é baseada nisso.
    
    \subsection{Transmissão por rádio}
        % o que é 
        
        \par
        A transmissão de dados por rádio pode acontecer de duas maneiras: não-direcional e direcional  \cite{torres2001}.
        
        \begin{itemize}
        \item{Não-Direcional: }
        Quando a transmissão é feita pela forma não-direcional, a frequencias são emitidads em todas as direções e qualquer antena localizada na região de alcance das ondas de rádio podem captar os dados \cite{torres2001}, isso pode ser visto na \autoref{fig:nao_direciona}.
            \begin{figure}[H]
              \caption{\label{fig:nao_direciona}{Transmissão não-direcional.}}
              \centering
              \includegraphics[width=0.5\textwidth]{Figuras/transmissao_radio_nao_direcional.PNG}
              \legend{Fonte: \cite{torres2001}}
            \end{figure}
        \par
        A  \autoref{fig:nao_direciona} mostra que as ondas de radio são emitidas em todas as direções e que qualquer antena/recepetor que estiver no alcance pode receber os dados do emissor.
        
        \item{Direcional: }
        A transmissão no sistema direcional necessita que os aparelhos transmissores e receptores estejam apontando um na direção do outro para que haja comunicação, sem contar que não pode ter obstaculos entre eles senão dificulta a transmissão \cite{torres2001}. A \autoref{fig:direcional} exemplifica o funcionamento dessa forma de transmissão.

        \begin{figure}[H]
              \caption{\label{fig:direcional}{Transmissão direcional.}}
              \centering
              \includegraphics[width=0.5\textwidth]{Figuras/transmissao_radio_direcional.PNG}
              \legend{Fonte: \cite{torres2001}}
        \end{figure}
        \par
        De outro modo a \autoref{fig:direcional} mostra que o primeiro transmissor dever estar direcionado para a direcção do segundo trasmissor, e assim dever acontecer com o segundo transmissor também.
        \end{itemize}    
\section{Localização}
\par
Segundo \citeauthor{rfid2009review}, as informações situacionais de pessoas ou objetos têm um papel muito importante nas aplicações, dessa forma é possível saber a posição dos objetos ou pessoas para assim fazer um monitoramento ou utilizar para várias outras aplicações. Essas informações podem ser obtida através de sistemas de posicionamento, que podem ser classificados dependendo do ambiente, podendo ser destinada a ambientes internos ou externo. Esta seção aborda alguns dos tipos de sistemas de localização.
    \subsection{GPS}
    \par
    GPS é um sistema de posicionamento global implementado pelo programa NAVSTAR \textit{(Navigation System Timing and Ranging)}, iniciado no ano de 1973. Era mantido pela divisão de sistema espacial dos Estados Unidos e era destinado apenas para uso militar \cite{gpsEduardo2005}.
    \par
   O principal objetivo do uso de GPS é determinar as coordenadas espaciais de pontos referentes a um sistema mundial, para isso o sistema faz uso de distâncias entre quatros satélites, a posição do receptor é calculada a partir dos sinais recebido pelos satélites \cite{gpsEduardo2005}.

   \begin{figure}[H]
              \caption{\label{fig:satelites}{Constelação de Satélites.}}
              \centering
              \includegraphics[width=0.5\textwidth]{Figuras/gps_satelites.PNG}
              \legend{Fonte: \cite{gpsEduardo2005}}
        \end{figure}
        \par
        A  \autoref{fig:satelites} mostra a movimentação dos satélites em torno da Terra para assim enviar sinais para os receptores que por sus vez interpretam esses sinais resultando em seu posicionamento na Terra.
 
    \subsection{WLAN}
    \par
    A localização em ambientes indoor utilizando WLANs pode ser feita com a RSSI, \textit{Angle of Arrival} (AOA), ou \textit{Time Difference of Arrival} (TDOA) \cite{wifiFernandes}. Os dispositivos devem possuir conectividade sem fio para que seja possível saber seu posicionamento no ambiente. A função que permite a utilização de RSSI está disponível em todas as interfaces 802.11 \cite{Wlan2012}.
    
    \par
    Entre as inúmeras maneiras de localizar dispositivos em ambientes fechados utilizando WLANs, algumas são: 
    \begin{itemize}
        \item {Triangulação}
        \par
        Essa forma de localizar faz uso de AOA, que seria computação dos ângulos a múltiplos ponto de acesso, o resultado disso é uma interceptação que resulta na provável localização, isso pode ser visto na \autoref{fig:triangulacao} \cite{wifiFernandes}. 
           \begin{figure}[H]
              \caption{\label{fig:triangulacao}{Triangulação.}}
              \centering
              \includegraphics[width=0.5\textwidth]{Figuras/triangulacao.PNG}
              \legend{Fonte: \cite{wifiFernandes}}
        \end{figure}
        \item {Trilateração: }
        \par 
       Utilizando propriedades geométricas essa técnica faz cálculos entre múltiplos pontos de acesso para assim obter a posição do dispositivo \cite{wifiFernandes}.
        \begin{figure}[H]
              \caption{\label{fig:trilateracao}{Trilateração.}}
              \centering
              \includegraphics[width=0.5\textwidth]{Figuras/trilateracao.PNG}
              \legend{Fonte: \cite{wifiFernandes}}
        \end{figure}
       \par
        Na \autoref{fig:trilateracao} é mostrado que há uma a comunicação entre os pontos de acesso para assim poder efetuar os cálculos, esses cálculos são uma forma de saber a TDOA para assim estimar a posição do objeto \cite{wifiFernandes}.
        
        \item {Reconhecimento de padrões }
        \par
        O RSSI é o principal requisito dessa técnica, em que é feita medições prévias para fazer uma comparação com os dados do banco de dados. Inicialmente é necessário uma fase em que é feita a calibração para se obter o mapa de assinatura \cite{wifiFernandes}.
        
        \par
        O mapa de assinatura é basicamente dados do banco que representam a coleção das medidas de RSSI em diferentes locais para todos os pontos de acesso no ambiente \cite{wifiFernandes}. Na \autoref{fig:fingerprinting} é mostrado seu funcionamento, onde cada ponto de acesso se comunica com o servidor para assim fazer uma comparação com os dados do mapa de assinatura.
         \begin{figure}[H]
              \caption{\label{fig:fingerprinting}{Reconhecimento de padrões.}}
              \centering
              \includegraphics[width=0.5\textwidth]{Figuras/fingerprinting.PNG}
              \legend{Fonte: \cite{wifiFernandes}}
        \end{figure}
    \end{itemize}

    \subsection{RFID}
    \par
    Nessa subseção estaremos mostrando a técnica de localização com RFID utilizada no sistema \textit{LocAlizatioD iDentification based on dynaMic Active Rfid Calibration} (LANDMARC) proposto por \citeauthor{landmarc}, visto que esse sistema é citado na grande maioria das fontes que retratam a utilização de RFID para localizar objetos em ambientes fechados.
    
    \par
    O LANDMARC faz uso de tags RFID ativas para determinar o local das tags que serão localizadas, essas tags ativas são colocadas em pontos já conhecidos pelo sistema e servem como pontos de referência e assim diminuir o número de leitores e ter uma melhor precisão no ambiente \cite{RFIDapplicationsTechniques}.
    
    \par
    Através das tags ativas é possível se obter informações com relação a intensidade do sinal, essa informação é utilizada para calibrar a distância para as tags de rastreio por meio de uma soma com o peso atribuído as tags de referências mais próximas, é importante resaltar que a precisão dos resultados depende da forma que as tags de referência são posicionadas \cite{RFIDapplicationsTechniques}.
    
    \begin{figure}[H]
              \caption{\label{fig:landmarc2a}{LANDMARC.}}
              \centering
              \includegraphics[width=0.4\textwidth]{Figuras/landmarc2a.png}
              \legend{Fonte: \cite{landmarc}}
        \end{figure}
    \par
    Na \autoref{fig:landmarc2a} mostra a aplicação sendo utilizada em um ambiente real por \citeauthor{landmarc}, em que os retangulos cinzas são leitores RFID, os retangulos brancos são tags de refências e os pontos pretos são tags a serem localizadas, e dessa forma o sistema calcula as coordenadas dos objetos.
\begin{comment}    
    \subsection{Sistemas de Localização }
     \begin{itemize}
        \item{LANDMARC}
        \item{RADAR}
        \item{SpotON}
     \end{itemize}
\end{comment}    
\section{Algoritmos de Localização indoor}
Nesta seção serão abordados alguns dos algoritmos que já foram utilizados para localização em ambientes indoor.
    \subsection{Multilateração}
    A multilaterção estima a coordenadas do nó de destino a partir das distâncias do nó de destino para o nó de refrência que possui coordenadas conhecidas, é o mesmo que ocorre na trilateração, porém a multilateração pode-se utilizar 2,3 ou n nós de referências 
    \subsection{Inferência Bayesiana}
    \subsection{Nearest-neighbor}
    \subsection{Proximidade}
    \subsection{Aprendizado Baseada em Kernel}
\section{Modelagem e Prototipação de Sistemas}
    \subsection{UML}
    \subsection{Redes de Petri}